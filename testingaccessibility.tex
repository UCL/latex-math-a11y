\documentclass[]{article}

\usepackage{graphicx}
\usepackage{pdfcomment}
\usepackage[svgnames]{xcolor}
\usepackage{setspace}
\usepackage{hyperref}
\usepackage{natbib}

  \title{LaTeX for absolute beginers teaching STEM subjects}
  \author{Jim Tyson}
  \date{\today}

\renewcommand{\familydefault}{\sfdefault}

\makeatletter
\DeclareRobustCommand{\em}{%
  \@nomath\em \if b\expandafter\@car\f@series\@nil
  \normalfont \else \bfseries \fi}
\makeatother

\providecommand{\tightlist}{%
  \setlength{\itemsep}{0pt}\setlength{\parskip}{0pt}}
\pagecolor{Ivory}

\begin{document}

\onehalfspacing

\maketitle
\tableofcontents

\section{What is \LaTeX{}?}

\subsection{Markup and WIMPs}

There are various ways of creating documents for reading in print, on screen or to be read.  People are often familiar with the more modern \emph{What You See Is What You Get} approach that is implemented in a word processor application like Microsoft Word.  In this approach, the user edits an on-screen approximation of what the final document will look like, updated in real time as they edit it.  Typically, the document is formatted using a \emph{Windows, Icons, Mouse and Pointer} interface, often using the \emph{Object-Verb} action scheme: select and highlight some text and then modify it.  This method is attractive to users and it is usually held to require little learning.  The truth is that using these applications effectively \emph{does} require learning, but using them effectively is uncommon.

The major alternative to this \emph{WIMP} method is the \emph{markup} approach.  In this scheme, a document is prepared with all content in plain text - just the characters you could enter using Windows notepad, for example - and the content is interspersed with coded instructions concerning formatting, structure and layout.  Examples of the markup approach include \emph{SGML, HTML \TeX{}} and \emph{\LaTeX{}} and more recent markup 'lite' approaches, such as \emph{markdown}.  These systems are commonly held to be difficult to learn and adapt to, although enthusiasts insist that the learning curve is relatively shallow and that the systems encourage good document preparation practices.

\subsection{The case for markup}

Even after committing ourselves to produce accessible and inclusive materials using common applications, there remain some domains where these applications do not provide adequate solutions.  Perhaps the most common is where there is a requirement to typeset anything but the simplest mathematics.  While Microsoft Word, for example, has a good equation editor, the program cannot easily produce PDFs with machine readable (text to speech) mathematical content and the typesetting is less than good.  Moreover, many creators in the STEM domains will already be familiar with - and often be enthusiastic users of - LaTeX for document preparation.

\subsection{\LaTeX{} for the uninitiated}

\LaTeX{} is a markup system devised by Leslie Lamport.  It is in fact a set of macros (simplifying commands) for the \TeX{} typesetting system devised by Donald Knuth, a mathematician and computer scientist who typeset his own noteworthy books on computation and mathematics.

A \LaTeX{} file is prepared as a file of plain text and commands.  This file is passed through the \emph{\LaTeX{} engine} and in principle a variety of output formats can be produced, but most commonly the output is a PDF.

\LaTeX{} can be used with a dedicated desktop editor and processor such as MiKTeX, TeXMaker, many of which run on Windows, Mac and Linux desktops.  Alternatively, there are online document production systems such as Overleav which provide an online edtor and processor to produce documents in the cloud.

Which ever system you use the core language and techniques are identical.

\subsubsection{Document classes and libraries}




\end{document}
